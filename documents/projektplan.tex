	% !TEX TS-program = pdflatex
	% !TEX encoding = UTF-8 Unicode
	\documentclass[a4paper]{article}
	\usepackage[swedish]{babel}
	\usepackage[T1]{fontenc}
	\usepackage[utf8]{inputenc}
	\usepackage[pdftex]{graphicx}
	\usepackage{float}
	\usepackage{fancyhdr}
	\usepackage{geometry}
	\usepackage{booktabs} % for much better looking tables
	\usepackage{array} % for better arrays (eg matrices) in maths
	\usepackage{paralist} % very flexible & customisable lists (eg. enumerate/itemize, etc.)
	\usepackage{verbatim} % adds environment for commenting out blocks of text & for better verbatim
	\usepackage{subfig} % make it possible to include more than one captioned figure/table in a single float
	
	%%% HEADERS & FOOTERS
	
	\author{Jonathan Karlsson - jonka293 - 890201-1991}
	\pagestyle{fancy} % options: empty , plain , fancy
	\renewcommand{\headrulewidth}{1pt} % customise the layout...
	\fancyhead[LO,LE]{Jonathan Karlsson\\Projektplan - TSBK03}
	\lfoot{}\cfoot{\thepage}\rfoot{}
	
	%%%% SECTION TITLE APPEARANCE
	
	%\usepackage{sectsty}
	%\allsectionsfont{\sffamily\mdseries\upshape} % (See the fntguide.pdf for font help)
	%% (This matches ConTeXt defaults)
	%
	%%%% ToC (table of contents) APPEARANCE
	%\usepackage[nottoc,notlof,notlot]{tocbibind} % Put the bibliography in the ToC
	%\usepackage[titles,subfigure]{tocloft} % Alter the style of the Table of Contents
	%\renewcommand{\cftsecfont}{\rmfamily\mdseries\upshape}
	%\renewcommand{\cftsecpagefont}{\rmfamily\mdseries\upshape} % No bold!
	
	%%% END Article customizations
	
	%%% The "real" document content comes below...
	
	\title{Projektplan - TSBK03}
	
	%\date{} % Activate to display a given date or no date (if empty),
	% otherwise the current date is printed 
	
	\begin{document}
	
	\maketitle

	\section{Idé}
	Jag vill göra ett projekt med en människa i 3D som springer runt i en värld med varierad höjd på marken. 
	\begin{itemize}
	\item Gubben ska vara animerad med hjälp av ben och begränsningar på benen. (Funderar på skinning med benstrukturer och rotationsgränser på varje ben.)
	\item Gubben ska kunna kollidera på ett bra sätt med andra objekt.
	\item Lagom mycket fysik ska vara implementerat. (Kanske fysikmotor).
	\item Ett visst mål med världen bör finnas, skjuta något, ta sig någonstans.
	\item Kontrollera gubben via tangentbord och mus.
	\item Skall vara snygg struktur på koden så det är enkelt att lägga till funktionallitet.
	\item Skall vara skrivet i c++.
	\end{itemize}

	\section{Om jag får tid}
	\begin{itemize}
	\item Gubben ska kasta en fin naturlig skugga med t.ex shadow mapping (kanske testa olika metoder).
	\item Lägga till så att solen går upp och ner för att simulera ett dygn.
	\item Lägga in skjutfunktioner för att döda saker (med poäng?).
	\item Testa gränser för hur många animerande objekt man kan ha.
	\item Fixa så att användaren kan kontrollera prestandan genom att stänga av funktionalitet.
	\item Använda geometry shader för att simulera gräs som rör sig i vinden med en sinusfunktion. 
	\end{itemize}

	\section{Milstolpe}
	Mitt i nästa period bör det finnas en värld med en gubbe som kan springa runt med animationer, kanske inte helt perfekta ännu men så det händer något. Även viss fysik bör finnas för att få animationerna att fungera bättre.
	
	\section{Gruppmedlemmar}
	\begin{itemize}
	\item Jonathan Karlsson
	\end{itemize}
	
	\section{Svårighetsgrad}
	I sitt grundstadie ganske lagom svårt men med de andra tilläggen kommer detta nog bli ganska komplicerat, speciellt att få allt att fungera tillsammans.
	
	\end{document}